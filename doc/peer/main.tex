\documentclass[]{llncs}

\usepackage{xspace}

\usepackage{color} 
\usepackage{graphicx} 
\usepackage{amssymb}
%\usepackage{amsthm} 
\usepackage{amsmath}

\DeclareMathOperator*{\argmin}{arg\,min}

\usepackage{listings} 
\usepackage{hyperref} 
\usepackage{systeme}

\usepackage[dvipsnames]{xcolor} 
\usepackage[colorinlistoftodos,prependcaption,textsize=tiny]{todonotes}
\newcommand{\bruno}[1]{\todo[linecolor=OliveGreen,backgroundcolor=OliveGreen!25,bordercolor=OliveGreen]{#1}\xspace}
\newcommand{\alex}[1]{\todo[linecolor=Red,backgroundcolor=Red!25,bordercolor=Red]{#1}\xspace}

\usepackage[T1]{fontenc}
\usepackage{dsfont}
\usepackage{textcomp}

\usepackage{xcolor}

\definecolor{dkgreen}{rgb}{0,0.6,0}

% Formatting for Configurable Parameters: 
% Argument 1: parameter's name 
% Argument 2: parameter's type
\newcommand{\conf}[2]{{\color{dkgreen}\texttt{#1:#2}}}
\newcommand{\param}[1]{{\color{dkgreen}\texttt{#1}}}

\begin{document}

\title{A Flexible Peer Management Architecture \\ for Blockchain Systems}

\author{Alexander Chepurnoy and Bruno Woltzenlogel Paleo} 

\maketitle

\begin{abstract}
For a blockchain network to be secure and efficient, 
it is essential that nodes follow a disciplined peer management strategy, 
trying to avoid malicious peers and seeking to discover 
and retain reliable honest peers.
This paper defines an abstract peer management architecture, 
relying on flexible ranking and clustering of peers. 
A concrete implementation of this architecture in the 
Scorex framework is also described.
\end{abstract}


\section{Introduction}

Blockchain systems have been increasingly adopted in recent years. 
For instance, the total market cap of Bitcoin \cite{TODO}, the most popular cryptocurrency 
and blockchain application, has grown to more than 130 billion US 
dollars\footnote{Data obtained from \url{http://www.worldcoinindex.com} in 2017 
on the 24th of November 2017.} since its inception in 2009. 
And Bitcoin is not an isolated phenomenon. Fifteen other cryptocurrencies 
have market caps larger than 1 billion US dollars. With so much at stake, 
the security of these systems is critical. Consensus 
protocols, such as Bitcoin's Proof of Work (PoW) \cite{TODO} 
and variants of Proof of Stake (PoS) \cite{TODO} 
(e.g. Cardano's Ouroboros \cite{TODO} and Ethereum's TODO \cite{TODO}), 
ensure that the communicating peers agree on a single blockchain and 
that the blockchain remains valid as long as honest nodes detain the majority 
of computation power or stake. However, the network itself is vulnerable to attacks
that may prevent the peers from communicating effectively. A recent study \cite{TODO}
found, for example, that Bitcoin's peer management is significantly vulnerable
to eclipse attacks, where an attacker running several nodes attempts to monopolize 
all incoming and outgoing connections of an honest node. If the attacker succeeds,
the mining power or stake of that honest node, is effectively removed from the network.
Furthermore, the attacker may fork the chain and lead the honest node 
to waste mining power (and profits) on the forked chain. The possibility of such 
network layer attacks highlights the importance of resilient peer management strategies 
and motivates the work presented here.

Bitcoin's rigid and hard-coded peer management strategy \cite{TODO} 
distributes known peers in two tables, one for new peers and 
the other for tried peers, and groups them in buckets so 
that peers belonging to the same subnet 
(and hence more likely to be controlled by an attacker) 
are more likely to belong to the same bucket. 
A bitcoin node then tries to connect to peers from different buckets
and from different tables.

The goal of the peer management architecture proposed here is 
to generalize Bitcoin's approach and make it more flexible, 
because there may be many other characteristics, 
besides IP address and being new or tried, that may be 
relevant for a node to consider 
when deciding whether to connect to a peer. 
The proposed architecture, described abstractly in 
Section \ref{TODO}, depends on a peer ranking algorithm 
(cf. Section \ref{TODO})
and a peer clustering algorithm (cf. Section \ref{TODO}); 
and procedures and protocols for manipulating peers 
(cf. Section \ref{TODO}), from peer banning to selection of 
peers for connection, may use the ranking and the clusters.
Finally, an implementation of the proposed architecture in the Scala-based 
Scorex framework for blockchains is presented in Section \ref{TODO}.




\section{Abstract Peer Management Systems}

\newcommand{\peers}{\mathcal{P}}

% Feature function
% Argument 1: peer
% Argument 2: feature name
\newcommand{\feature}[2]{\phi(#1, #2)}
\newcommand{\featureFunction}{\phi}
\newcommand{\features}{\mathcal{F}}

\newcommand{\rank}{\rho_{\featureFunction}}
\newcommand{\clusters}{\xi_{\featureFunction}}

\newcommand{\nat}{\mathds{N}\xspace}

A \emph{peer} is a pair $(\mathrm{addr}, \mathrm{port})$, 
where $\mathrm{addr}$ is its IP address and $\mathrm{port}$ is its port. 
The set of all peers is denoted $\peers$. Any characteristic of a peer 
(e.g. number of past connections, date of last connection, reputation, delivery delays) 
is a \emph{feature} and the finite set of all relevant features is denoted $\features$.
\alex{TODO: do you think we need to separate ingoing / outgoing connections?}


A \emph{peer management structure} is a tuple $(G, B, C)$, 
where $G$ is the set of \emph{good peers}, $B$ is the set of \emph{banned peers}, and 
$C$ is the set of \emph{connected peers}, satisfying the following conditions: 
$G \cap B = \emptyset$, 
$C \subseteq G$,
$G \subseteq \peers$, 
$B \subseteq \peers$.
%
A \emph{peer management system} is a tuple $(M, \featureFunction, \rank, \clusters, \Pi)$ where:
\begin{itemize}
\item $M$ is a peer management structure; 
%
\item $\featureFunction$ is the \emph{feature map} such that $\feature{p}{f}$ 
is the value of feature $f$ for peer $p$;
%
\item $\rank: \forall S: 2^{\peers}. S \rightarrow \nat$ \bruno{I'm using a dependent type notation here, in a sloppy way. This could be improved.} 
 is a \emph{peer ranking} such that $\rank(S, p)$ is the \emph{rank} of peer $p$ in the set of peers $S$ using the feature map $\featureFunction$;
%
\item $\clusters: 2^{\peers} \rightarrow 2^{2^{\peers}}$ is a \emph{peer clusterizer} using the feature map $\featureFunction$ that takes a set of peers $S$ 
and returns a set of \emph{clusters} $\clusters(S) \equiv \{ S_1, \ldots, S_m \}$ such that 
$S_i \cap S_j = \emptyset$ when $i \neq j$ and $\bigcup_{1 \leq i \leq m} S_i = S$;
%
\item $\Pi$ is a set of \emph{protocols} and \emph{procedures} that take the current $M$ and the current feature map $\featureFunction$ and, possibly using $\rank$ and $\clusters$, 
return a modified peer management structure $M'$ or a modified feature map $\featureFunction'$.
\end{itemize} 

\begin{example}
Examples of peer rankings include: a function that sorts the set of peers $S$ according to any combination of features in $\phi$ 
and returns the index of $p$ in the sorted ordered set; or a function that 
returns the number of days since the last connection with $p$.
\end{example}

\begin{example}
Bitcoin's usage of separate tables for \emph{new} peers, with whom there has never been a connection, and already \emph{tried} peers, 
can be seen as a simple and rigid example of clusterizer. Bitcoin's bucket system, which partition peers according to their subnets, 
can also be seen as a clusterizer.\bruno{TODO: More examples, not related to bitcoin, here.}
\end{example}


\section{A Concrete Peer Ranking}
\label{sec:PeerRanking}


\newcommand{\indexOf}{\mathrm{indexOf}}
\newcommand{\sortBy}{\mathrm{sort}}
\newcommand{\isIncreasing}{\mathrm{isIncreasing}}

The peer ranking function $\rank$ is defined by:
\[
\rank(S, p) \equiv \sum_{f \in F_r} (\indexOf(p, \sortBy(S, f, \isIncreasing(f)))) * \conf{\$\{f\}Weight}{Int}
\]
where:
\begin{itemize}
\item $F_r$ is the set of features relevant for ranking, which has 
      \texttt{averageDeliveryDelay}, 
      \texttt{numberOfPastConnections} and 
      \texttt{elapsedTimeSinceLastConnection} as elements.
\item $\indexOf(p, L)$ is the index of $p$ in a list $L$.
\item $\sortBy(S, f, b)$ is the list of elements in $S$ sorted by their values on feature $f$. 
      Sorting is increasing if $b = \texttt{true}$, and decreasing otherwise.
\item $\isIncreasing(f)$ is $\texttt{true}$ iff the sorting of peers according to feature $f$ 
      should be increasing. $\isIncreasing$ is $\texttt{true}$ for 
      \texttt{numberOfPastConnections} and \texttt{elapsedTimeSinceLastConnection} and 
      $\texttt{false}$ for \texttt{averageDeliveryDelay}.
\item For each feature $f$, $\conf{\$\{f\}Weight}{Int}$ is a configurable parameter 
      indicating the importance of feature $f$ for the ranking.
\end{itemize}

Peers having a higher rank are considered to be better than peers having a lower rank. 
By sorting peers decreasingly with respect to their average delivery delay, 
faster peers are prioritized.
By sorting peers increasingly with respect to their number of past connections, 
reliable peers that have already been tried and used many times are preferred.
By sorting peers increasingly with respected to their elapsed time since last connection,
rotation of peers is encouraged.


\section{A Concrete Clusterizer}
\label{sec:Clusterizer}


\newcommand{\kmeans}{\texttt{k-means\textbardbl}}

The peer clusterizer $\clusters$ is defined by:
\[
\clusters(S) 
\equiv 
\kmeans_{\param{numberOfClusters}}(\mathrm{map}(\lambda p. \mathrm{toVector}^{\featureFunction}_{F_c}(p)*\param{weightVector}, S))
\]
\bruno{TODO: There is a minor typing issue here that I need to fix. 
       kmeans is returning clusters of vectors, 
       but it should be returning clusters of peers corresponding to those vectors.}
where:
\begin{itemize}
\item $\kmeans$ is the parallel scalable k-means++ 
      clustering algorithm~\cite{parallelKMeans}%http://vldb.org/pvldb/vol5/p622_bahmanbahmani_vldb2012.pdf
\item \param{numberOfClusters} is the configurable parameter specifying 
      how many clusters should be generated.
      It defaults to \param{maxConnections}.
\item $\mathrm{toVector}^{\featureFunction}_{F_c}(p)$ transforms a peer $p$ 
      into its vector of feature values $\mathrm{map}(\lambda f. \feature{p}{f}, F_c)$.
\item $F_c$ is the vector of features relevant for clustering, which has
      \texttt{ipOctetOne}, \texttt{ipOctetTwo}, \texttt{ipOctetThree}, \texttt{ipOctetFour},
      \bruno{dividing an IP address into 4 dimensions in these 4 octets takes care of clustering in the same cluster/bucket 
      peers that belong to the same subnet when IP allocation follows the old \emph{classfull network} scheme. 
      It probably works well for \emph{variable-length subnet masking} as well. 
      But, in this case, maybe something better could be done if we knew the routing prefix. 
      But I think we don't. Or do we? More info: \url{https://en.wikipedia.org/wiki/Classless_Inter-Domain_Routing}}
      \texttt{numberOfPastConnections}.
\item \param{weightVector} is the vector of weights $\mathrm{map}(\lambda f. \param{\$\{f\}Weight}, F_c)$.
\end{itemize}

By having IP octets as features relevant for clustering, peers that belong to the same subnet
will tend to be in the same cluster. Since the subnet masking convention considers the leftmost
bits of an IP address to be more significant (i.e. the more leftmost bits are shared 
by two IP addresses, the more likely they are to belong to the same subnet), 
it should be ensured that 
$\texttt{ipOctetOneWeight} >> \texttt{ipOctetTwoWeight} >> \texttt{ipOctetThreeWeight} >> \texttt{ipOctetFourWeight}$,
where $>>$ means ``significantly greater than''.

By taking \texttt{numberOfPastConnections} into account, peers that differ in
how known or new they are will tend to belong to different clusters. 



\section{Peer Management Protocols and Procedures} 
\label{sec:PeerManagementProtocols}

Peers are inserted and removed from the sets of good peers, banned peers and connected peers
through protocols and procedures. The following subsections describe the concrete protocols 
and procedures that are\bruno{will be} implemented in Scorex.


\subsection{Peer Discovery}

A peer discovery protocol requests new peers from a source and inserts them into
the set of good peers $G$. Generally, a source may be another peer, a
trusted central server or an untrusted communication channel (e.g. IRC, Twitter, \ldots).

In Scorex, when the size of $G$ is
below \conf{minGoodPeers}{Int}, the peer requests new peers from $p^* = \argmin_p \rank(G, p)$. 
$p^*$ replies by sending a set of new peers $N$. 
$G$ is then updated to $G' \equiv G \cup (N \setminus B)$. 

\alex{TODO: have "referencedBy" column in the peers table}
\bruno{monitor and record “number of incoming connections” and 
“number of outgoing connections” in the feature table. 
Then it is up to the ranking and clustering functions to use this information as they wish. }

\bruno{TODO: request from a central server? download from an untrusted communication channel?}


\subsection{Peer Selection}


The peer selection protocol is executed whenever the size of $C$ is smaller 
than \conf{minConnections}{Int}. Its goal is to select a subset $N$ of 
$n \equiv \conf{maxConnections}{Int} - |C|$ 
peers from $G \setminus C$ and then update $C$ to $C' \equiv C \cup N$.


This is achieved through the following steps:
\begin{enumerate}
\item Partition $G$ into the clusters $\{ G_1, \ldots, G_m \} \equiv \clusters(G)$. 
\item Initialize $C'$ with $C$. 
\item For each cluster $G_k$, let $h(G_k) \equiv |C' \cup G_k|$.
\item Then, for $i$ from $1$ to $n$:
   \begin{enumerate}
   \item Let $H = \{ G_k \in \clusters(G) | h(G_k) = \min_{g \in \clusters(G)} h(g) \}$
   \item Randomly choose a cluster $G^*$ from $H$ with uniform probability.
   \item Randomly choose an integer $r$ in the $[0, 99]$ integer interval and then choose $p^*$ from $G^*$ with one of the following probability distributions (for all $p \in G^*$):
      \begin{itemize}
      	 \item $P(p) \equiv 1/|G^*|$, if $r < \conf{peerExplorationProbability}{Int}$.
         \item $P(p) \equiv (\rank(G^*, p))/(\sum_{q \in G^*} \rank(G^*, q)) G^*$, otherwise.
      \end{itemize}
   \item Let $C' \equiv C' \cup \{ p^* \}$.      
   \end{enumerate}
\end{enumerate}



Finally, the occasional selection of peers with uniform probability, irrespectively of their rank,
gives a chance of connection even to new peers that might have a low rank.


As the peer selection procedure chooses to connect to a set of peers that belong to different clusters,
it will tend to choose a mix of old and new peers that belong to different subnets.



\subsection{Feeler Connections}

Bitcoin does the following:
        //  * Choose a random address from new and attempt to connect to it if we can connect
        //    successfully it is added to tried.
        //  * Start attempting feeler connections only after node finishes making outbound
        //    connections.


\subsection{Handshake}

During a handshake protocol a peer tries to get information 
on the availability of another peer and its capabilities. 
For example, if another peer is working in SPV mode, 
so does not have full blocks, they should not being asked from it. 
Or if the peer is interested in encrypted traffic exchange only, 
it would like to refuse. Encryption key is also to be formed during the handshaking. 

\alex{key to be agreed on during handshaking 
(via elliptic curve Diffie-Hellman key exchange); 
then symmetric encryption}

\alex{there are couple of standards how to use symmetric crypto with the key agreed on; ECIES-KEM / PSEC-KEM}



\subsection{Good Peer Eviction}

Removes peers from \textbf{GoodPeers} when its number of peers exceeds
\conf{maxGoodPeers}{Int}.

TODO: use ranking and cluster

\subsection{Banned Peer Eviction}

Removes peers from \textbf{BannedPeers} when its number of peers
exceeds \conf{maxBannedPeers}{Int}.

TODO: use ranking and cluster


\subsection{Peer Banning}

Moves a peer from \textbf{GoodPeers} to \textbf{BannedPeers}.







\subsection{Peer Rotation}

Replaces a peer in \textbf{ConnectedPeers} by another peer from
\textbf{GoodPeers}.

Speeds up recovery from eclipse attacks.




\section{Information Exchange Protocols}

\section{Encryption and Authentication}

\alex{ECIES, AES, MAC}


\section{Configuration Advice}


In order to increase the peer-to-peer protocol's resilience against
eclipse attacks by a group of malicious peers that pretend to be good
until the moment when they decide to attack, \conf{maxGoodPeers}{Int}
should be set sufficiently high, in order to keep the proportion of
malicious peers in \textbf{GoodPeers} low compared to the proportion
of honest peers. Then the probability that \emph{all} connections are
made with malicious peers, as required for an eclipse attack to
succeed, is low as well.

As long as we do not want to ever forgive banned peers\bruno{Would we
ever want to forgive banned peers?}, \conf{maxBannedPeers}{Int} should
be set as high as possible without exceeding the available
memory\bruno{I'm assuming that we want to load \textbf{BannedPeers}
entirely into memory to be able to check quickly whether an incoming
connection request is from a banned peer. Is this assumption
correct?}.





\section{Implementation in Scorex}

In order to allow a high degree of flexibility, the proposed peer
management architecture has many configurable parameters. Throughout
the paper, configurable parameters and their types are formatted as
``\conf{anExampleParameter}{ItsType}''.

TODO: Parallel implementation of clustering % (see reference 13 here: http://vldb.org/pvldb/vol5/p622_bahmanbahmani_vldb2012.pdf)


The peer management system maintains persistent and mutually disjoint
tables of \textbf{GoodPeers}\footnote{In Bitcoin, presumably good
peers are split in two tables: \textbf{New} and \textbf{Tried}. Here a
new peer is simply a good peer with zero past connections.} and
\textbf{BannedPeers}. There are at most \conf{maxGoodPeers}{Int} peers
in \textbf{GoodPeers} and at most \conf{maxBannedPeers}{Int} in
\textbf{BannedPeers}.

\begin{example}
TODO: Show an example table
\end{example}

The peer management system maintains a non-persistent table of
\textbf{ConnectedPeers} to which the peer is currently connected.
There are at most \conf{maxConnections}{Int} peers in
\textbf{ConnectedPeers}.

\section{Conclusion}


Future work: reputation system

Future work: Anomaly detection: Consider a peer anomalous if it is the only element of a
cluster?


\bibliographystyle{alpha} 
\bibliography{bibliography}


\end{document}
